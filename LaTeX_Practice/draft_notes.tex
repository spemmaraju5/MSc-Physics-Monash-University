\documentclass{article}
\usepackage[utf8]{inputenc}

\title{\textbf{Background Knowledge: Axion Like Particles (ALPs)}}
\author{Sai Pemmaraju}
\date{April 2022}

\begin{document}

\maketitle

\section{Axions}
\subsection{Strong CP Problem and the Motivation Behind Axions}
We seek to determine why quantum chromodynamics (QCD) seems to preserve CP-symmetry. In the mathematical formulation of QCD, CP (charge-parity) symmetry can be violated in strong interactions.
However, no violation of this symmetry has been observed experimentally. There is no known reason in QCD for it to necessarily be conserved, and as a result, this is a "fine-tuning" problem\footnote{The process by which parameters of a model must be adjusted very precisely in order to fit with certain observations}\\
\\
An \textbf{axion} is a hypothetical elementary particle proposed to 
resolve the strong CP problem. The interaction Lagrangian of QCD is given by:
\begin{eqnarray}
    \mathcal{L}_{QCD} = -\frac{1}{4}G_{\mu\nu}G^{\mu\nu}-\frac{g_{s}^{2}\theta}{32\pi^{2}}G_{\mu\nu}\tilde{G}^{\mu\nu}+\bar{\psi}(i\gamma^{\mu}D_{\mu}-me^{i\theta'\gamma_{5}})\psi\\
    = \frac{\theta_{QCD}}{32\pi^{2}} Tr G_{\mu\nu}\tilde{G^{\mu\nu}}
\end{eqnarray}
In the above equations $G$ refers to the gluon field strength tensor, $\tilde{G^{\mu\nu}} = \epsilon^{\alpha\beta\mu\nu}G_{\alpha\beta}$ is the dual and the trace runs over the colour $SU(3)$ indices
In Equation (1), the terms $G_{\mu\nu}\tilde{G}^{\mu\nu}$ and $i\theta'$ are CP violating. Furthermore, $\theta$ and $\theta'$ can be interpreted as angles (i.e. $\theta, \theta'\in[0,2\pi))$). These terms can be combined to form a total effective angle $\bar{\theta}$
Since there is no expermiental evidence of CP violation in the strong interaction, this implies that $|\bar{\theta}|\approx 0$. In an attempt to resolve this problem, Peccei and Quinn (1977) proposed a solution that involves promoting $\bar{\theta}$ to a field through the addition of 
a new global symmetry (Peccei-Quinn (PQ) symmetry) which is spontaneously broken, which results in a new particle which has been named as the "axion". The introduction of such a particle implies that the CP violating terms will be set to zero without the need for any fine tuning.\\
\\
Five experiments have been conducted in an attempt to search for evidence of axions. However, all of these have yielded negative results
\subsection{Axion-Like Particles (ALPs)}
The key difference between axions and Axion-Like Particles (ALPs), which result from the aforementioned breaking of the PQ symmetry is that the latter are not as constrained as axions (i.e. their masses and coupling strengths are independent parameters). 
In addition to this, ALPs couple predominantly to gauge bosons (specifically pairs of bosons, such as $gg, \gamma\gamma, ZZ, W^{+}W^{-}, \gamma Z$)
\end{document}


 
