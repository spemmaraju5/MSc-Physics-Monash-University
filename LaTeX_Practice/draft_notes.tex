\documentclass{article}
\usepackage[utf8]{inputenc}
\usepackage[compat=1.0.0]{tikz-feynman}

\title{\textbf{Background Knowledge: Search for Axion Like Particles (ALPs) at the BABAR Experiment}}
\author{Sai Pemmaraju}
\date{April 2022}

\begin{document}

\maketitle
\section{Introduction}
The analysis aims to search for an Axion Like Particle in the decay $B^{\pm}\rightarrow K^{\pm}+A, A\rightarrow \gamma\gamma$
The ALP that are searched for and presented in the analysis would be produced in flavour changing neutral current processes\footnote{Interactions that change the flavour of a fermion without altering its electric charge}), since these are rare in the Standard Model, and present good prospects of
a signature that is likely to lead to discovery. In addition to this, the theory does not present a constraint on the mass of an ALP, and the search is therefore conducted up to the kinematical limit of the decay, given by:
$$m^{max}_{A} = m_{B^{\pm}}-m_{K^{\pm}} = 4786 MeV/c^{2}$$
The analysis aims to reconstruct the decay of a charged B-meson into a charged $K$ and two photons, and to search resonances in the diphoton invariant mass spectrum. Various multivariate analuses have been tested to discriminate the signal from the background, and different selection strategies have been accounted for to reject the two main backgrounds
(i.e. the one related to light quark production (i.e.$uds$ and $c$ quarks). Shape variables, which describe the topology of the event, are necessary to reject the $uds$ background. Since the B-like background is similar to the signal, it is more complex to reject these.\\
Hence, two BDTs\footnote{Boosted Decision Trees} are optimised to reject each of the different sources of background. Twelve variables have been used, including shape variables, kinematic variables, etc. is preceeded by a rough pre-selection, which is finalised to reduce the quantity of data to process without losing signal events, to train the selection on a reduced and simplified sample\\
\\
The selection has been trained on a signal sample produced with Monte Carlo simulation and modified to have a flat diphoton mass distribution in the considered range\footnote{This is done to avoid the introduction of biases in the training selection proceedure, caused by the overspecialisation over a specific signal ALP mass like the energy of the two photons}

\section{Axions}
\subsection{Strong CP Problem and the Motivation Behind Axions}
We seek to determine why quantum chromodynamics (QCD) seems to preserve CP-symmetry. In the mathematical formulation of QCD, CP (charge-parity) symmetry can be violated in strong interactions.
However, no violation of this symmetry has been observed experimentally. There is no known reason in QCD for it to necessarily be conserved, and as a result, this is a "fine-tuning" problem\footnote{The process by which parameters of a model must be adjusted very precisely in order to fit with certain observations}\\
\\
An \textbf{axion} is a hypothetical elementary particle proposed to 
resolve the strong CP problem. The interaction Lagrangian of QCD is given by:
\begin{eqnarray}
    \mathcal{L}_{QCD} = -\frac{1}{4}G_{\mu\nu}G^{\mu\nu}-\frac{g_{s}^{2}\theta}{32\pi^{2}}G_{\mu\nu}\tilde{G}^{\mu\nu}+\bar{\psi}(i\gamma^{\mu}D_{\mu}-me^{i\theta'\gamma_{5}})\psi\\
    = \frac{\theta_{QCD}}{32\pi^{2}} Tr G_{\mu\nu}\tilde{G^{\mu\nu}}
\end{eqnarray}
In the above equations $G$ refers to the gluon field strength tensor, $\tilde{G^{\mu\nu}} = \epsilon^{\alpha\beta\mu\nu}G_{\alpha\beta}$ is the dual and the trace runs over the colour $SU(3)$ indices
In Equation (1), the terms $G_{\mu\nu}\tilde{G}^{\mu\nu}$ and $i\theta'$ are CP violating. Furthermore, $\theta$ and $\theta'$ can be interpreted as angles (i.e. $\theta, \theta'\in[0,2\pi))$). These terms can be combined to form a total effective angle $\bar{\theta}$
Since there is no expermiental evidence of CP violation in the strong interaction, this implies that $|\bar{\theta}|\approx 0$. In an attempt to resolve this problem, Peccei and Quinn (1977) proposed a solution that involves promoting $\bar{\theta}$ to a field through the addition of 
a new global symmetry (Peccei-Quinn (PQ) symmetry) which is spontaneously broken, which results in a new particle which has been named as the "axion". The introduction of such a particle implies that the CP violating terms will be set to zero without the need for any fine tuning.\\
\\
Five experiments have been conducted in an attempt to search for evidence of axions. However, all of these have yielded negative results
\subsection{Axion-Like Particles (ALPs)}
The key difference between axions and Axion-Like Particles (ALPs), which result from the aforementioned breaking of the PQ symmetry is that the latter are not as constrained as axions (i.e. their masses and coupling strengths are independent parameters). 
In addition to this, ALPs couple predominantly to gauge bosons (specifically pairs of bosons, such as $gg, \gamma\gamma, ZZ, W^{+}W^{-}, \gamma Z$). In other words, an ALP does not intend to solve the strong CP problem, but its phenomenology is identical to that of an axion. These ALPs are ideal Dark Matter candidates

\section{B-Meson Decay Process}
The following decay is considered for the subsequent sections\footnote{A B-meson is one that consists of a $b\bar{b}$ pair}
\begin{equation}
    B\rightarrow K^{*}A, A\rightarrow\gamma\gamma
\end{equation}
The coupling of the ALPs to weak gauge bosons (i.e. the $W^{\pm}$) leads to observable signatures. The effective Lagrangian is given by:
\begin{equation}
    \mathcal{L} = (\partial_{\mu}a)^{2}-\frac{1}{2}m_{a}^{2}a^{2}-\frac{g_{aW}}{4}W_{\mu\nu}\bar{W}^{\mu\nu}
\end{equation}


One can observe the decay $B\rightarrow K^{*}A, A\rightarrow\gamma\gamma$ using B-factories which are desgined to generate and observe B-decays

\subsection{B-Factories}
B factories were constructed with the intention of testing the CKM (Cabibo-Kobayashi-Maskawa) description of quark mixing and CP violation in the Standard Model. These detectors performed precise measurements of the CKM matrix elements and of several branching ratios of rare B-meson decays. In order to achieve these goals, a B-factory 
is required to achieve a high luminosity\footnote{The luminosity is proportional to the number of collisions that occur in a given amount of time. The greater the luminosity, the more data the experiments can gather to observe rare processes}. It is also mandatory to boost the produced $B\bar{B}$ pairs to increase their respective displacement 
in order to distinguish between them. \\
\\
The PEP-II 

\section{Analysis Overview and Data Samples}
The search is only conducted in the charged mode $B^{\pm}\rightarrow K^{\pm}A$ with an un-excited $K$ (since the neutral mode has higher background, and the sensitivity is significantly reduced as a result). Reconstructing the $B^{\pm}\rightarrow K^{\pm}\gamma\gamma$ constrains the ALP mass to be lower than: $$m_{A}^{max} = m_{B_{\pm}}-m_{K} = 5279.26-493.68 MeV/c^{2} = 4785.58 MeV/c^{2}$$
The following steps can be adhered to to describe the characteristics of both the signal and the background:
\begin{itemize}
    \item Event selection (discriminating between the signal and the background)
    \item Verifying the optimal event selection for consistency by checking against a real data control sample to check that the Monte Carlo (MC) simulation is a good model of the actual data
    \item Performing a fit to extract the signal yielded
    \item Estimating systematic errors
\end{itemize}

\subsection{Event Selection}
The following variables are used to discriminate between the signal and the background:
\begin{itemize}
    \item $m_{ES}$, which represents the beam energy substituted mass of the B-meson candidate
    \item $\Delta E$, which represents the energy difference between the B-meson and the beam in the centre of mass frame
    \item Helicity angle of $K$
    \item Thrust angle
    \item Sphericity angle
    \item Legendre Moments
    \item Maximum Selector of $K$
    \item Photon Veto, which is a variable that is designed to reduce contamination from real $\pi^{0},$, $\eta$ and $\eta'$
    \item Photon Energies
\end{itemize}


\end{document}





