\documentclass[11.5pt]{report}

\usepackage[a4paper,width=160mm,top=25mm,bottom=25mm,bindingoffset=6mm]{geometry}
\usepackage{times}
\usepackage{lipsum}
\usepackage{titlesec}

\titlespacing\section{0pt}{12pt plus 3pt minus 1pt}{0pt plus 1pt minus 1pt}
\titlespacing\subsection{0pt}{12pt plus 3pt minus 1pt}{0pt plus 1pt minus 1pt}
%headers and footers
\usepackage{fancyhdr}
\pagestyle{fancy}

\usepackage[T1]{fontenc}
\usepackage{ae,aecompl}
\usepackage{natbib}
\usepackage{float}
\usepackage{placeins}

% Only include extra packages if you really need them. Common packages are:
\usepackage[utf8]{inputenc}
\usepackage{graphicx}	% Including figure files
\usepackage{amsmath}	% Advanced maths commands
\usepackage{amssymb}	% Extra maths symbols

\usepackage{xspace}
\usepackage{caption}
\usepackage{subcaption}
\usepackage{caption,setspace}
%%%%%%%%%%%%%%%%%%%%%%%%%%%%%%%%%%%%%%%%%%%%%%%%%%

%%%%% AUTHORS - PLACE YOUR OWN COMMANDS HERE %%%%%

% Please keep new commands to a minimum, and use \newcommand not \def to avoid
% overwriting existing commands. Example:
%\newcommand{\pcm}{\,cm$^{-2}$}	% per cm-squared

\newcommand{\ergcm}[1]{$\times 10^{#1}$ erg cm$^{-2}$ s$^{-1}$}
\newcommand{\oergcm}[1]{$10^{#1}$ erg cm$^{-2}$ s$^{-1}$}
\newcommand{\ergs}[1]{$\times 10^{#1}$ erg s$^{-1}$}
\newcommand{\oergs}[1]{$10^{#1}$ erg s$^{-1}$}
\newcommand{\expo}[1]{$\times 10^{#1}$}
\newcommand{\oexpo}[1]{$10^{#1}$}
\newcommand{\kms}{km s$^{-1}$}
\newcommand{\msun}{M\textsubscript{\(\odot\)}}
%\newcommand{\msun}{M$_{\odot}$}

%%%%%%%%%%%%%%%%%%% TITLE PAGE %%%%%%%%%%%%%%%%%%%

\begin{document}
\label{firstpage}
%\pagerange{\pageref{firstpage}--\pageref{lastpage}}

\begin{titlepage}
    \begin{center}
        
        \vspace{1.5cm}
        \includegraphics[width=0.65\textwidth]{monashlogo.png}
        
        \large
        {School of Physics and Astronomy/Astrophysics}
        
        \vspace{2.5cm}
        \Large
        {MASTERS THESIS}
        
        \huge
        \line(1,0){250}\\
        \textbf{The Search for Axion Like Particles (ALPs) Through $B$ Meson Decays at the LHCb}\\
        \line(1,0){250}
        
        \vspace{2.0cm}
        \huge
        {Subrahmanya Saicharan Pemmaraju}\\
        \Large
        {ID: 32734719}
        
        \vspace{1.5cm}
        \huge
        {Supervised by: Prof.Ulrik Egede}
        
        \vspace{5.5cm}
        \Large
        {Date: \today}
        
    \end{center}
\end{titlepage}

\pagenumbering{arabic}

%#################################################
\newpage
\chapter*{Abstract}
% Abstract here 

%#################################################
\newpage
\chapter*{Acknowledgements}
I would first like to convey my heartfelt and sincere gratitude to my parents, and my younger brother Piyush, who have supported me every step of the way throughout this arduous, yet highly rewarding journey. Their unwavering faith in my capabilities during the tough times is unparalleled and I am eternally grateful to them for enabling me to pursue this body of work. 
Secondly, I would like to thank my supervisor, Prof. Ulrik Egede, without whose support and guidance, this body of work would not have taken its present form. I am eternally grateful to Prof. Egede for his continual support throughout the years and all of the additional opportunities that he has presented me to become more involved with the collaboration, and to further explore the realm of experimental particle physics. \\
\\

%#################################################
\newpage
\tableofcontents

%%%%%%%%%%%%%%%%% BODY OF REVIEW %%%%%%%%%%%%%%%%%%
\setlength{\parskip}{1em}
\renewcommand{\baselinestretch}{1.5}

\newpage
\chapter{Introduction}
\chapter{Background and Motivation}
% Enter Chapter 1 content here
% Basics of the Standard Model go here, include limitations of the model
\section{Synopsis of the Standard Model}
% and hint to strong CP problem here
\section{The Strong CP Problem}The two discrete symmetries that are essential to the motivation of the Strong CP problem are charge conjugation, $C$, and parity (i.e. an inversion of spatial coordinates), $P$. While each
of these symmetries can be individually violated by various physical phenomena, their combination CP is known to be conserved in both the strong and electromagnetic interactions, whilst being violated by weak interactions. The strong CP problem arises from the theory pertaining to QCD, which
permits such a violation. Despite this, however, such a process has not been experimentally observed. One can examine the QCD Lagrangian in Equation \ref{QCD_Lagrangian} below, which has been written to include the CP violating terms
\begin{equation}\label{QCD_Lagrangian}
    \mathcal{L}_{QCD} = -\frac{1}{4}G_{\mu\nu}G^{\mu\nu}-\frac{g_{s}^{2}\theta}{32\pi^{2}}G_{\mu\nu}\tilde{G}^{\mu\nu}+\bar{\psi}(i\gamma^{\mu}D_{\mu}-me^{i\theta'\gamma_{5}})\psi
\end{equation}
The terms


% Describe the Strong CP problem here and hint towards it's resolution
\subsection{Axions}
\subsection{Experimental Searches for Axions}
\subsection{Axion Like Particles (ALPs)}
\subsection{The $B\rightarrow K^{*}A, A\rightarrow\gamma\gamma$ Decay Process}

\chapter{The LHCb Detector}
The LHCb detector is a flavour physics experiment at the Large Hadron Collider (LHC) that aims to reconstruct particles 
consisting of $c$ and $b$ quarks. It is dedicated to precision measurements of CP violation and rare decays of the aforementioned $b$ 
hadrons.  
\section{Structure of the LHCb Detector}
The LHCb is a single-arm spectrometer with a forward angular coverage from approximately 300 (250) mrad in the bending (non-bending) plane. The detector is designed 
\subsection{Vertex Locator (VELO)}
\subsection{Ring Imaging Cherenkov (RICH) Detector}
\subsection{Magnet}
A dipole magnet is used to measure the momentum of charged particles by exploiting the curvature of their trajectories in the presence of a magnetic field. The measurement covers the forward acceptance of $\pm$ 250 mrad vertically and of $\pm$300 mrad horizontally. The magnet possesses an integrated magnetic field of 4 Tm for tracks of 10 m length, and is designed to accomodate the varying requirements of the magnetic field, ranging from 2 mT within the RICH envelope, and a maximum value in the regions between the Vertex Locator and the Trigger Tracker tracking station.\\
\\
The magnet possesses saddle-shaped coils in a window frame yoke with sloping poles in order to match the required detector acceptance
\subsection{Calorimeters}
\subsubsection{HCAL}
HCAL is awesome 
\subsubsection{ECAL}
ECAL is even more awesome
\section{Data Analysis at the LHCb}
\subsection{The LHCb Data Flow}
\subsection{Software Overview}
\subsection{The LHCb Simulation Framework}




\chapter{Experimental Methods}
\input{Chapter_3.tex}
\chapter{Results}
\chapter{Discussion}
\


% Can add more Chapters just follow the same idea above. 
% If you want a different name for the chapter in the header edit the above line as follows \chapter[New Name]{Official Chapter Name}

%##########################################################
\newpage
\chapter*{Conclusion}

% Add Conclusion here

\addcontentsline{toc}{chapter}{Conclusion}
\addcontentsline{toc}{chapter}{References}
%%%%%%%%%%%%%%%%%%%% REFERENCES %%%%%%%%%%%%%%%%%%

% The best way to enter references is to use BibTeX:
%\bibliography{example} % if your bibtex file is called example.bib
\footnotesize
% uncomment and change to the bibiliography style you prefer
%\bibliographystyle{apalike}
%\bibliography{refs}

%%%%%%%%%%%%%%%%% APPENDICES %%%%%%%%%%%%%%%%%%%%%

%\appendix

%\section{Some extra material}

%If you want to present additional material which would interrupt the flow of the main paper,
%it can be placed in an Appendix which appears after the list of references.

%%%%%%%%%%%%%%%%%%%%%%%%%%%%%%%%%%%%%%%%%%%%%%%%%%

% Don't change these lines
%\bsp	% typesetting comment

\label{lastpage}
\bibliographystyle{plain} % We choose the "plain" reference style
\bibliography{refs}
\end{document}
