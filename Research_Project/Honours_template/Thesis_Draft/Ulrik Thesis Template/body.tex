\section{Introduction}
\label{sec:Introduction}
 
This is the template for typesetting LHCb notes and journal papers.
It should be used for any document in LHCb~\cite{LHCb-DP-2008-001} that is to be
publicly available. The format should be used for uploading to
preprint servers and only afterwards should specific typesetting
required for journals or conference proceedings be applied. The main
\latex file contains several options as described in the \latex comment
lines.

%%%%%%%%%%%%%%%%%%%%%%%%%%%%%%%%%%%%%%%%%%%%%%%%%%%%%%%%%%%%%%%%%%%%%%%%%%%%%%%%%%%%%%%%%%%%%%%%%
\section{General principles}

The main goal is for a paper to be clear. It should be as brief as
possible, without sacrificing clarity. For all public documents,
special consideration should be given to the fact that the reader will
be less familiar with \lhcb than the author.

Here follow a list of general principles that should be adhered to:
\begin{enumerate}

\item Choices that are made concerning layout and typography
  should be consistently applied throughout the document.

\item Standard English should be used (British rather than American)
  for LHCb notes and preprints. Examples: colour, flavour, centre,
  metre, modelled and aluminium. Words ending on -ise or -isation
  (polarise, hadronisation) can be written with -ize or -ization ending but should be consistent.
  The punctuation normally follows the closing quote mark of quoted text, 
  rather than being included before the closing quote.
  Footnotes come after punctuation. 
  Papers to be submitted to an American journal can be written in American
  English instead. Under no circumstance should the two be mixed.

\item Use of jargon should be avoided where possible. ``Systematics'' are ``systematic
  uncertainties'', ``L0'' is ``hardware trigger'', Monte-Carlo'' is ``simulation'', ``penguin'' diagrams
  are best introduced with an expression like ``electroweak loop (penguin) diagrams'', ``cuts'' are ``selection requirements''. The word ``error'' is ambiguous as it can mean the difference between the true and measured values or your estimate thereof. The same applies to event, that we usually take to mean the whole $pp$ collision; candidate or decay can be used instead.''

\item It would be good to avoid using quantities that are internal jargon and/or 
  are impossible to reproduce without the full simulation, \ie\ instead of ``It is 
  required that $\chisqvtx<3$'', to say ``A good quality vertex is required''; 
  instead of ``It is required that $\chisqip>16$'', to say ``The track is 
  inconsistent with originating from a PV''; instead of ``A DLL greater 
  than 20 is required'' say to ``Tracks are required to be identified as kaons''.
  However, experience shows that some journal referees ask for exactly this 
  kind of information, and to safeguard against this, one may consider given 
  some of it in the paper, since even if the exact meaning may be LHCb-specific, 
  it still conveys some qualitative feeling for the significance levels required 
  in the varies steps of the analysis.   
 
\item The abstract should be concise, and not include citations or
  numbered equations, and should give the key results from the paper.

\item References should usually be made only to publicly accessible
  documents. References to LHCb conference reports and public notes
  should be avoided in journal publications, instead including the
  relevant material in the paper itself.

\item The use of tenses should be consistent. It is recommended to
  mainly stay in the present tense, for the abstract, the description
  of the analysis, \etc; the past tense is then used where necessary,
  for example when describing the data taking conditions.

\item A sentence should not start with a variable, a particle or an acronym.
 A title or caption should not start with an article. 

\item Incorrect punctuation around conjunctive adverbs and the use of 
dangling modifiers are the two most common mistakes of English grammar
in LHCb draft papers. If in doubt, read the wikipedia articles on 
conjunctive adverb and dangling modifier.  

\item When using natural units, at the first occurrence of an energy unit
      that refers to momentum or a radius, add a footnote: ``Natural units
      with $\hbar=c=1$ are used throughout.'' Do this even when somewhere
      a length is reported in units of mm. It's not 100\% consistent, but
      most likely nobody will notice. The problem can be trivially avoided
      when no lengths scales in natural units occur, by omitting the 
      $\hbar$ from the footnote text.

\item Papers dealing with amplitude analyses and/or resonance parameters, 
      other than masses and lifetimes, should use natural units,
      since in these kind of measurements widths are traditionally 
      expressed in MeV and radii in GeV$^{-1}$. It's also the convention
      used by the PDG. 

\end{enumerate}

%%%%%%%%%%%%%%%%%%%%%%%%%%%%%%%%%%%%%%%%%%%%%%%%%%%%%%%%%%%%%%%%%%%%%%%%%%%%%%%%%%%%%%%%%%%%%%%%%
\section{Layout}

\begin{enumerate}

\item Unnecessary blank space should be avoided, between paragraphs or
  around figures and tables.

\item Figure and table captions should be concise and use a somewhat smaller typeface
  than the main text, to help distinguish them. This is achieved by 
  inserting \verb!\small! at the beginning of the caption.
  (NB with the latest version of the file \verb!preamble.tex! this is automatic)
  Figure captions go below the figure, table captions go above the
  table.

\item Captions and footnotes should be punctuated correctly, like
  normal text. The use of too many footnotes should be avoided:
  typically they are used for giving commercial details of companies,
  or standard items like coordinate system definition or the implicit
  inclusion of charge-conjugate processes.\footnote{If placed at the end
    of a sentence, the footnote symbol normally follows the
    punctuation; if placed in the middle of an equation, take care to
    avoid any possible confusion with an index.}$^,$\footnote{The standard footnote reads: ``The inclusion of charge-conjugate processes is implied
    throughout.'' This may need to be modified, for example with ``except in the discussion of asymmetries.''}$^,$\footnote{The LHCb coordinate system is right-handed, with the $z$ axis pointing along the
beam axis, $y$ the vertical direction, and $x$ the horizontal direction. The $(x, z)$ plane is the bending plane
of the dipole magnet.}

\item Tables should be formatted in a simple fashion, without
  excessive use of horizontal and vertical lines.
  Numbers should be vertically aligned on the decimal point and $\pm$ symbol.
  (\verb!\phantom{0}! may help, or defining column separators as \verb!@{\:$\pm$\:}!)
  See Table~\ref{tab:example} for an example. 


\begin{table}[t]
  \caption{
    %\small %captions should be a little bit smaller than main text
    Background-to-signal ratio estimated in a $\pm 50\mevcc$ 
    mass window for the prompt and long-lived background sources, and the 
    minimum bias rate. In this table, as the comparison of numbers among columns
  is not critical, the value $11\pm2$ may also be typeset without the space.}
\begin{center}\begin{tabular}{lr@{\:$\pm$\:}lr@{\:$\pm$\:}ll}
    \hline
    Channel                           & \multicolumn{2}{c}{$B_{\mathrm{pr}}/S$} & \multicolumn{2}{c}{$B_{\mathrm{LL}}/S$}   & MB rate       \\ 
    \hline
    \BsToJPsiPhi              & $ 1.6$ &$0.6$ & $0.51 $ & $ 0.08$ & $\sim 0.3$ Hz \\
    \BdToJPsiKst              & $ 11\phantom{.0}$ & $ 2$ &  $1.5\phantom{0}$ & $ 0.1 $ & $\sim 8.1$ Hz \\
    \decay{\Bp}{\jpsi\Kstarp} & $ 1.6 $ & $ 0.2$ & $0.29 $ & $ 0.06$  & $\sim 1.4$ Hz \\
    \hline
  \end{tabular}\end{center}
\label{tab:example}
\end{table}

\item Figures and tables should normally be placed so that they appear
  on the same page as their first reference, but at the top or bottom
  of the page; if this is not possible, they should come as soon as
  possible afterwards.  They must all be referred to from the text.

\item If one or more equations are referenced, all equations should be numbered using parentheses as shown in
  Eq.~\ref{eq:CKM},
  \begin{equation}
    \label{eq:CKM}
    \Vus\Vubs + 
    \Vcs\Vcbs + 
    \Vts\Vtbs = 0 \ . 
  \end{equation}
  
\item Displayed results like
  \begin{equation*}
    \BF(\decay{\Bs}{\mumu}) < 1.5 \times 10^{-8} \text{ at 95\% CL}
  \end{equation*}
  should in general not be numbered.

\item Numbered equations should be avoided in captions and footnotes.

\item Displayed equations are part of the normal grammar of the
  text. This means that the equation should end in full stop or comma if
  required when reading aloud. The line after the equation should only
  be indented if it starts a new paragraph.

\item Equations in text should be put between a single pair of \$ signs.
  \verb!\mbox{...}! ensures they are not split over several lines.
  So \mbox{$\epsilon_\text{trigger}=(93.9\pm0.2)\%$} is written as
  \verb!\mbox{$\epsilon_\text{trigger}=(93.9\pm0.2)\%$}! and not as
  \verb!$\epsilon_\text{trigger}$=(93.9$\pm$0.2)\%! which generates the oddly-spaced $\epsilon_\text{trigger}$=(93.9$\pm$0.2)\%.

\item Sub-sectioning should not be excessive: sections with more than three
levels of index (1.1.1) should be avoided.

%\item It is generally preferable to itemize a list using numbers rather
%than bullets.

\item Acronyms should be defined the first time they are used,
  \eg ``A dedicated boosted decision tree~(BDT) is designed to select doubly Cabibbo-suppressed~(DCS) decays.''
  The abbreviated words should not be capitalised if it is not naturally
  written with capitals, \eg quantum chromodynamics (QCD),
  impact parameter (IP), boosted decision tree (BDT).
  Avoid acronyms if they are used three times or less.
  A sentence should never start with an acronym and its better to
  avoid it as the last word of a sentence as well.

\end{enumerate}

%%%%%%%%%%%%%%%%%%%%%%%%%%%%%%%%%%%%%%%%%%%%%%%%%%%%%%%%%%%%%%%%%%%%%%%%%%%%%%%%%%%%%%%%%%%%%%%%%
\section{Typography}
\label{sec:typography}

The use of the \latex typesetting symbols defined in the file
\texttt{lhcb-symbols-def.tex} and detailed in the appendices of this
document is strongly encouraged as it will make it much easier to
follow the recommendation set out below.

\begin{enumerate}

\item \lhcb is typeset with a normal (roman) lowercase b.

\item Titles are in bold face, and usually only the first word is
  capitalised.

\item Mathematical symbols and particle names should also be typeset
  in bold when appearing in titles.

\item Units are in roman type, except for constants such as $c$ or $h$
  that are italic: \gev, \gevcc.  The unit should be separated from
  the value with a thin space (``\verb!\,!''),
  and they should not be broken over two lines.
  Correct spacing is automatic when using predefined units inside math mode: \verb!$3.0\gev$! $\to 3.0\gev$.
  Spacing goes wrong when using predefined units outside math mode AND forcing extra space:
  \verb!3.0\,\gev! $\to$ 3.0\,\gev or worse:   \verb!3.0~\gev! $\to$ 3.0~\gev. 

\item  If factors of $c$ are kept, they should be used both for masses and
  momenta, \eg $p=5.2\gevc$ (or $\gev c^{-1}$), $m = 3.1\gevcc$ (or $\gev c^{-2}$). If they are dropped this
  should be done consistently throughout, and a note should be added
  at the first instance to indicate that units are taken with $c=1$.
  Note that there is no consensus on whether decay widths $\Gamma$ are
  in \mev or \mevcc (the former is more common).
  Both are accepted if consistent. 

\item The \% sign should not be separated from the number that precedes it: 5\%, not 5 \%. 
A thin space is also acceptable: 5\,\%, but should be applied consistently throughout the paper.

\item Ranges should be formatted consistently. The recommended form is to use a dash with no spacing around it: 
7--8\gev, obtained as \verb!7--8\gev!. Another possibility is ``7 to 8\gev''.

\item Italic is preferred for particle names (although roman is
  acceptable, if applied consistently throughout).  Particle Data
  Group conventions should generally be followed: \Bd (no need for a
  ``d'' subscript), \decay{\Bs}{\jpsi\phi}, \Bsb,
  (note the long bar, obtained with \verb!\overline!, in contrast to the discouraged short \verb!\bar{B}! resulting in $\bar{B}$), \KS (note the
  uppercase roman type ``S''). 
  This is most easily achieved by using the predefined symbols described in 
  Appendix~\ref{sec:listofsymbols}.

  Italic is also used for particles whose name is an uppercase Greek letter:
  \Upsilonres, \Deltares, \Xires, \Lambdares, \Sigmares, \Omegares, typeset as
  \verb!\Upsilonres!, \verb!\Deltares!, \verb!\Xires!, \verb!\Lambdares!, \verb!\Sigmares!, \verb!\Omegares! (or with the appropriate macros adding charge and subscripts). Paper titles in the bibliography must be adapted accordingly.
  Note that the \Lz baryon has no zero, while the \Lb baryon has one. That's historical.
  
\item Unless there is a good reason not to, the charge of a particle should be
  specified if there is any possible ambiguity 
  ($m(\Kp\Km)$ instead of $m(KK)$, which could refer to neutral kaons).


\item Decay chains can be written in several ways, depending on the complexity and the number of times it occurs. Unless there is a good reason not to, usage of a particular type should be consistent within the paper.
Examples are: 
\decay{\Dsp}{\phi\pip}, with \decay{\phi}{\Kp\Km}; 
\decay{\Dsp}{\phi\pip} (\decay{\phi}{\Kp\Km});  
\decay{\Dsp}{\phi(}{\Kp\Km)\pip}; or
\decay{\Dsp}{[\Kp\Km]_\phi\pip}.

\item Variables are usually italic: $V$ is a voltage (variable), while
  1 V is a volt (unit). Also in combined expressions: $Q$-value, $z$-scale, $R$-parity \etc

\item Subscripts and superscripts are roman type when they refer to a word (such as T
  for transverse) and italic when they refer to a variable (such as
  $t$ for time): \pt, \dms, $t_{\mathrm{rec}}$.

%  This is a test: $f_{\pt}$.

\item Standard function names are in roman type: \eg $\cos$, $\sin$
  and $\exp$.

\item Figure, Section, Equation, Chapter and Reference should be
  abbreviated as Fig., Sect. (or alternatively Sec.), Eq., Chap.\ and
  Ref.\ respectively, when they refer to a particular (numbered) item,
  except when they start a sentence. Table and Appendix are not
  abbreviated.  The plural form of abbreviation keeps the point after
  the s, \eg Figs.~1 and~2. Equations may be referred to either with 
  (``Eq.~(1)'') or without (``Eq.~1'') parentheses, 
  but it should be consistent within the paper.

\item Common abbreviations derived from Latin such as ``for example''
  (\eg), ``in other words'' (\ie), ``and so forth'' (\etc), ``and
  others'' (\etal), ``versus'' (\vs) can be used, with the typography
  shown, but not excessively; other more esoteric abbreviations should be avoided.
  

\item Units, material and particle names are usually lower case if
  spelled out, but often capitalised if abbreviated: amps (A), gauss
  (G), lead (Pb), silicon (Si), kaon (\kaon), but proton (\proton).

%\item The prefix for 1000 (k, \eg kV) should not be confused with
%  that used in computing (K, which strictly speaking denotes $2^{10}$,
%  \eg KB).

\item Counting numbers are usually written in words if they start a
  sentence or if they have a value of ten or below in descriptive
  text (\ie not including figure numbers such as ``Fig.\ 4'', or
  values followed by a unit such as ``4\,cm'').
  The word 'unity' can be useful to express the special meaning of
  the number one in expressions such as: 
``The BDT output takes values between zero and unity''.
% Numbers should not be
%  written as words if they by nature are real numbers that happen to
%  take an integer value, such as $\chisq/\mathrm{ndf} < 4$.

\item Numbers larger than 9999 have a small space between
  the multiples of thousand: \eg 10\,000 or 12\,345\,678.  The decimal
  point is indicated with a point rather than a comma: \eg 3.141.

\item We apply the rounding rules of the
  PDG~\cite{PDG2020}. The basic rule states that if the three
  highest order digits of the uncertainty lie between 100 and 354, we round
  to two significant digits. If they lie between 355 and 949, we round
  to one significant digit. Finally, if they lie between 950 and 999,
  we round up and keep two significant digits. In all cases,
  the central value is given with a precision that matches that of the
  uncertainty. So, for example, the result $0.827 \pm 0.119$ should be
  written as $0.83\pm 0.12$, $0.827\pm 0.367$ should turn into
  $0.8\pm 0.4$, and $14.674\pm0.964$ becomes $14.7\pm1.0$.
  When writing numbers with uncertainty components from
  different sources, \ie statistical and systematic uncertainties, the rule
  applies to the uncertainty with the best precision, so $0.827\pm
  0.367\stat\pm 0.179\syst$ goes to $0.83\pm 0.37\stat\pm 0.18\syst$ and
  $8.943\pm 0.123\stat\pm 0.995\syst$ goes to $8.94\pm 0.12\stat\pm
  1.00\syst$.

\item When rounding numbers, it should be avoided to pad with zeroes
  at the end. So $51237 \pm 4561$ should be rounded as $(5.12 \pm 0.46)
  \times 10^4$ rather than $51200 \pm 4600$. Zeroes are accepted for yields.

\item When rounding numbers in a table, some variation of the rounding
  rules above may be required to achieve uniformity.

\item Hyphenation should be used where necessary to avoid ambiguity,
  but not excessively. For example: ``big-toothed fish'' (to indicate
  that big refers to the teeth, not to the fish), but ``big white
  fish''.  A compound modifier often requires hyphenation
  (\CP-violating observables, \bquark-hadron decays, final-state
  radiation, second-order polynomial), even if the same combination in
  an adjective-noun combination does not (direct \CP violation, heavy
  \bquark hadrons, charmless final state).  Adverb-adjective
  combinations are not hyphenated if the adverb ends with 'ly':
  oppositely charged pions, kinematically similar decay.  Words
  beginning with ``all-'', ``cross-'', ``ex-'' and ``self-'' are
  hyphenated \eg\ cross-section and cross-check. ``two-dimensional''
  is hyphenated. Words beginning with small prefixes (like ``anti'',
  ``bi'', ``co'', ``contra'', ``counter'', ``de'', ``extra'', ``infra'',
  ``inter'', ``intra'', ``micro'', ``mid'', ``mis'', ``multi'', ``non'', ``over'',
  ``peri'', ``post'', ``pre'', ``pro'', ``proto'', ``pseudo'', ``re'', ``semi'',
  ``sub'', ``super'', ``supra'', ``trans'', ``tri'', ``ultra'', ``un'', ``under'' and
  ``whole'') are single words and should not be hyphenated
  \eg\ semileptonic, pseudorapidity, pseudoexperiment, multivariate,
  multidimensional, reweighted,\footnote{Note that we write weighted unless it's the second weighting} preselection, nonresonant, nonzero,
  nonparametric, nonrelativistic, antiparticle,
  misreconstructed and misidentified.

\item Minus signs should be in a proper font ($-1$), not just hyphens
  (-1); this applies to figure labels as well as the body of the text.
  In \latex, use math mode (between \verb!$$!'s) or make a dash (``\verb!--!'').
  In ROOT, use \verb!#minus! to get a normal-sized minus sign. 

\item Inverted commas (around a title, for example) should be a
  matching set of left- and right-handed pairs: ``Title''. The use of
  these should be avoided where possible.

\item Single symbols are preferred for variables in equations, \eg\
  \BF\ rather than BF for a branching fraction.

\item Parentheses are not usually required around a value and its
  uncertainty, before the unit, unless there is possible ambiguity: so
  \mbox{$\dms = 20 \pm 2\invps$} does not need parentheses,
  whereas \mbox{$f_d = (40 \pm 4)$\%} or \mbox{$x=(1.7\pm0.3)\times 10^{-6}$} does.
  The unit does not need to be repeated in
  expressions like \mbox{$1.2 < E < 2.4\gev$}.

\item The same number of decimal places should be given for all values
  in any one expression (\eg \mbox{$5.20 < m_B < 5.34\gevcc$}).

\item Apostrophes are best avoided for abbreviations: if the abbreviated term
  is capitalised or otherwise easily identified then the plural can simply add
  an s, otherwise it is best to rephrase: \eg HPDs, pions, rather
  than HPD's, \piz's, $\pion$s.

\item Particle labels, decay descriptors and mathematical functions are not nouns, and need often to be followed by a noun. 
Thus ``background from \decay{\Bd}{\pip\pim} decays'' instead of ``background from \decay{\Bd}{\pip\pim}'',
and ``the width of the Gaussian function'' instead of ``the width of the Gaussian''.

\item In equations with multidimensional integrations or differentiations, the differential terms should be separated by a thin space and the $\rm d$ should be in roman.
Thus $\int f(x,y) {\rm d}x\,{\rm d}y$ instead $\int f(x,y) {\rm d}x{\rm d}y$ and
$\frac{{\rm d}^2\Gamma}{{\rm d}x\,{\rm d}Q^2}$ instead of $\frac{{\rm d}^2\Gamma}{{\rm d}x{\rm d}Q^2}$.
% The d's are allowed in either roman or italic font, but should be consistent throughout the paper.

\item Double-barrelled names are typeset with a hyphen (\verb!-!), as in Gell-Mann, but joined named use an n-dash (\verb!--!), as in Breit--Wigner. 

\item Avoid gendered words. Mother is rarely needed. Daughter can be a decay product or a final-state particle. Bachelor can be replaced by companion.
\end{enumerate}


%%%%%%%%%%%%%%%%%%%%%%%%%%%%%%%%%%%%%%%%%%%%%%%%%%%%%%%%%%%%%%%%%%%%%%%%%%%%%%%%%%%%%%%%%%%%%%%%%
\section{Figures}
\label{sec:Figures}


\begin{enumerate}
\item Before you make a figure you should ask yourself what message
  you want to get across. You don't make a plot ``because you can''
  but because it is the best illustration of your argument. 
  
\item Figures should be legible at the size they will appear in the
  publication, with suitable line width.  Their axes should be
  labelled, and have suitable units (e.g. avoid a mass plot with
  labels in \mevcc if the region of interest covers a few \gevcc
  and all the numbers then run together).  Spurious background shading
  and boxes around text should be avoided.

\item For the $y$-axis, ``Entries'' or ``Candidates'' is appropriate in case no
background subtraction has been applied. Otherwise ``Yield'' or ``Decays''
may be more appropriate. If the unit on the $y$-axis corresponds to 
the yield per bin, indicate so, for example ``Entries / (5\mevcc)'' or ``Entries per 5\mevcc''.


\item Fit curves should not obscure the data points, and
   data points are best (re)drawn over the fit curves. In this
   case avoid in the caption the term ``overlaid'' when
   referring to a fit curve, and instead use the words  
   ``shown'' or ``drawn''.

\item Figures with more than one part should have the parts labelled
  (a), (b) \etc, with a corresponding description in the caption;
  alternatively they should be clearly referred to by their position,
  e.g. Fig.~1\,(left). In the caption, the labels (a), (b) \etc should
  precede their description. When referencing specific sub-figures,
  use ``see Fig. 1(a)'' or ``see Figs. 2(b)-(e)''.

\item Keep captions short. They should contain the information necessary to understand the figure, but no more. For instance the fit model does not need to be repeated. Describe the data first, then mention the fit components.

\end{enumerate}

%%%%%%%%%%%%%%%%%%%%%%%%%%%%%%%%%%%%%%%%%%%%%%%%%%%%%%%%%%%%%%%%%%%%%%%%%%%%%%%%%%%%%%%%%%%%%%%%%
\section{References}
\label{sec:References}

References should be made using Bib\TeX~\cite{BibTeX}. A special style
\texttt{LHCb.bst} has been created to achieve a uniform
style. Independent of the journal the paper is submitted to, the
preprint should be created using this style. Where arXiv numbers
exist, these should be added even for published articles. In the PDF
file, hyperlinks will be created to both the arXiv and the published
version, using the {\tt doi} for the latter.

Results from other experiments should be cited even if not yet published. 

\begin{enumerate}

\item Citations are marked using square brackets, and the
  corresponding references should be typeset using Bib\TeX\ and the
  official \lhcb Bib\TeX\ style. 

\item For references with four or less authors all of the authors'
  names are listed~\cite{Lee:1967iu}, otherwise the first author
  is given, followed by \etal. The \lhcb Bib\TeX\ style will
  take care of this. The limit of four names can be changed by changing the number 4 in 
  ``{\tt \#4 'max.num.names.before.forced.et.al :=}''
  in {\tt LHCb.bst}, as was done in Ref.~\cite{LHCb-PAPER-2017-038}.

\item The order of references should be sequential when reading the
  document. This is automatic when using Bib\TeX.

\item The titles of papers should in general be included. To remove
  them, change \texttt{\textbackslash
    setboolean\{articletitles\}\{false\}} to \texttt{true} at the top
  of this template.

\item Whenever possible, use references from the supplied files
\verb!main.bib!, \verb!LHCb-PAPER.bib!, \verb!LHCb-CONF.bib!, and \verb!LHCB-DP.bib!.
These are kept up-to-date by the EB. If you see a mistake, do not edit these files,
but let the EB know. This way, for every update of the paper, you save
yourself the work of updating the references. Instead, you can just copy or
check in the latest versions of the \verb!.bib! files from the repository.
{\bf Do not take these references from \texttt{inspirehep} instead} (``{\tt Aaaij:20XXxyz}''), as \texttt{inspirehep} sometimes adds mistakes, does not handle errata properly and does not use LHCb-specific macros.

\item For those references not provided by the EB, the best
  is to copy the Bib\TeX\ entry directly from
  \href{http://inspirehep.net}{inspirehep}.
    Often these need to be edited to get the 
  correct title, author names and formatting. The warning about special UTF8 characters should never be ignored. It usually signals a accentuated character in an author name.
  For authors with multiple initials, add a space between them (change \texttt{R.G.C.} to \texttt{R. G. C.}),
  otherwise only the first initial will be taken. 
  Also, make sure to eliminate unnecessary capitalisation.
  Apart from that, the title should be respected as much as possible
  (\eg do not change particle names to PDG convention nor introduce/remove factors of $c$, but do change Greek capital letters to use our slanted font.).
  Check that both the arXiv and the journal index are clickable
  and point to the right article.

%\item Even if the basic formatting of the Bib\TeX\ entry is taken from
%  \texttt{Inspire}, all the data should be cross checked against the
%  journal. Often there are minor changes to author initials or
%  titles. In case of a difference between the preprint and the
%  journal, the bibliographic information from the journal should be
%  used.

\item The \texttt{mciteplus}~\cite{mciteplus} package is used
  to enable multiple references to show up as a single item in the
  reference list. As an example \texttt{\textbackslash
    cite\{Cabibbo:1963yz,*Kobayashi:1973fv\}} where the \texttt{*}
  indicates that the reference should be merged with the previous
  one. The result of this can be seen in
  Ref.~\cite{Cabibbo:1963yz,*Kobayashi:1973fv}. Be aware that the
  \texttt{mciteplus} package should be included as the very last item
  before the \texttt{\textbackslash begin\{document\}} to work
  correctly.

\item It should be avoided to make references to public notes and
  conference reports in public documents. Exceptions can be discussed
  on a case-by-case basis with the review committee for the
  analysis. In internal reports they are of course welcome and can be
  referenced as seen in Ref.~\cite{LHCb-CONF-2012-013} using the
  \texttt{lhcbreport} category. For conference reports, omit the
  author field completely in the Bib\TeX\ record.

\item To get the typesetting and hyperlinks correct for \lhcb reports,
  the category \texttt{lhcbreport} should be used in the Bib\TeX\
  file. See Refs.~\cite{LHCb-INT-2011-047, *LHCb-ANA-2011-078,
    *CERN-THESIS-2014-057, *LHCb-PROC-2014-017, *LHCb-TALK-2014-257}
  for some examples. It can be used for \lhcb documents in the series
  \texttt{CONF}, \texttt{PAPER}, \texttt{PROC}, \texttt{THESIS},
  \texttt{LHCC}, \texttt{TDR} and internal \lhcb reports. Papers sent
  for publication, but not published yet, should be referred with
  their \texttt{arXiv} number, so the \texttt{PAPER} category should
  only be used in the rare case of a forward reference to a paper.

\item Proceedings can be used for references to items such as the
  \lhcb simulation~\cite{LHCb-PROC-2011-006}, where we do not yet have
  a published paper.

\end{enumerate}

There is a set of standard references to be used in \lhcb that are
listed in Appendix~\ref{sec:StandardReferences}.
