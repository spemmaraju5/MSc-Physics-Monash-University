\documentclass{article}
\usepackage[utf8]{inputenc}
\title{MSc Research Project Reading Notes}
\author{Sai Pemmaraju}
\date{April 2022}

\begin{document}

\maketitle
\section{New Experimental Approaches in the Search for Axion-Like Particles (Irastorza and Redondo (2018))}
\subsection*{Theoretical Motivation to Search for Axions}
Consider Lagrangian of SM at energies below EW symmetry breaking. There are two possible terms that violate parity (P) and time reversal without changing quark flavour
\begin{equation}
    \mathcal{L}_{CP} = -(\overline{\textbf{q}_{L}}m_{q}e^{i\theta_{\gamma}}\textbf{q}_{R}+ h.c)-\frac{\alpha_{S}}{8\pi}G^{a}_{\mu\nu}\tilde{G}^{\mu\nu}_{a}\theta_{QCD}
\end{equation}
where $\textbf{q} = (u, d, ...)$ is a vector of quark flavours, $\alpha_{S}$ is the QCD equivalent of the fine structure constant, and $G_{\mu\nu}$ is the gluon field strength tensor,
with $\tilde{G}^{\mu\nu^{a}} = \epsilon^{\nu\mu\alpha\beta}G_{\alpha\beta}^{a}/2$ is its dual, and, $\theta_{QCD}$ is the angle determining the gauge-invariant QCD vacuum. The measurement of the neutron EDM imposes the restriction that
$$|\theta| < 1.3\times 10^{-10}$$
where $\theta = \theta_{QCD}+N_{f}\theta_{Y}$, where $N_{f}$ represents the number of quark flavours and $\theta_{Y}$ is a common phase among these flavours. The essence of the strong CP problem is to determine why $\theta$ is so small if composed of two arbitrary phases.\\
\\
Peccei and Quinn noted that having a $U(1)_{A}$ symmetry only violated by the colour anomaly term $G\tilde{G}$ would clear the strong CP problem. They suggested that such a symmetry could exist if it were spontaneously broken at a high-energy scale. This symmetry is known as the Peccei Quinn (PQ) symmetry. 
Weinberg and Wilczek both realised that such a spontaneously broken global symmetry implied a new pseudo Nambu Goldstone (pNG) boson, which was named the 'axion'. The nature of the axion makes it an ideal dark matter candidate

\subsection*{Axion Like Particles (ALPs)}
Axion-like particles (ALPs) are particles with properties similar to those of axions. A key difference that exists between them is that both the mass $m_{A}$ and the coupling strength to photons $g_{a\gamma}$ are proportional to the energy scale
in the case of axions, whereas they are independent parameters for ALPs. ALPs cannot account for the strong CP problem, as they are much less constrained, while the axion is linked only to the strong force. ALPs can couple predominantly to gauge boson pairs such as $gg, \gamma\gamma, ZZ, \gamma Z, W^{+}W^{-}$, etc.

\subsection*{The $B\rightarrow K^{(*)}A, A\rightarrow \gamma\gamma$ Process}
The coupling of the ALP to the weak gauge bosons $W^{\pm}$ gives rise to observable signatures, unlike different ALP models wherein the main effective coupling is with photons and/or gluons (see Shuve, Lin paper) and the ALP mass is below 5 GeV. In the sitution where the ALP only directly couples with quarks, the aforementioned decay channel is only dominant below the $\pi$ mass threshold. Assuming that the ALP couples only to the field strength of the $SU(2)_{W}$ gauge bosons
\begin{equation}
    \mathcal{L} = (\partial_{\mu} a)^{2}-\frac{1}{2}M_{a}^{2}a^{2}-\frac{g_{aW}}{4}W_{\mu\nu}^{a}\tilde{W}^{a\mu\nu}
\end{equation}
\end{document}
